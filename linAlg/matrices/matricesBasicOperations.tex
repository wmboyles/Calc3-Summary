\section{Basic Operations}

\subsection{Vector Space Operations}
Two matrices are considered equal if they have the same number of rows and columns and all entries are equal.
Scalar multiplication works by multiplying each element by the scalar.
Addition works element by element.

\begin{example}
	\begin{align*}
		A &= \begin{bmatrix}
			1 & 2 \\
			-1 & 3
		\end{bmatrix} \text{, } B = \begin{bmatrix}
			4 & 0 \\
			-2 & 5
		\end{bmatrix}. \\
		A + B &= \begin{bmatrix}
			1 + 4 & 2 + 0 \\
			-1 + -2 & 3 + 5
		\end{bmatrix} = \begin{bmatrix}
			5 & 2 \\
			-3 & 8
		\end{bmatrix} \\
		3A &= \begin{bmatrix}
			3\cdot1 & 3\cdot 2 \\
			3\cdot -1 & 3\cdot 3
		\end{bmatrix} = \begin{bmatrix}
			3 & 6 \\
			-3 & 9
		\end{bmatrix}.
	\end{align*}
\end{example}

Notice addition is commutative, associative, has an additive identity (the all 0's matrix), and has an additive inverse (scalar multiply by -1).
Further, scalar multiplication has a multiplicative identity (1), is distributive over both addition and field multiplication.
Thus, the set of all matrices of the same size form a vector space.
We denote the vector space of all $m \times n$ matrices with real entries as $\mathcal{M}_{m \times n}$.

\subsection{Multiplication}
We can also define an operation for multiplying two matrices of compatible sizes that outputs another matrix.

\begin{definition}
	Let $A$ be an $m \times n$ matrix, and let $B$ be and $n \times k$ matrix.
	Then $C = AB$ is an $m \times k$ matrix where
	\begin{equation*}
		C_{i,j} = \sum_{d=1}^{n}{A_{i,d} B_{d,j}}.
	\end{equation*}
\end{definition}
If you're familiar with the concept of dot products, then $C_{i,j}$ is the dot product of the $i$th row of $A$ with the $j$th column of $B$.

\begin{example}
	\begin{align*}
		A &= \begin{bmatrix}
			1 & 2 & 3 \\
			0 & -1 & 2
		\end{bmatrix} \text{, } B = \begin{bmatrix}
			1 & -1 & 0 & 2 \\
			2 & 0 & 3 & -1 \\
			0 & 1 & 1 & 5 
		\end{bmatrix}. \\
		AB &= \begin{bmatrix}
			5 & 2 & 9 & 15 \\
			-2 & 2 & -1 & 11
		\end{bmatrix}.
	\end{align*}
\end{example}

Similar to scalar multiplication, there exists a multiplicative identity matrix.
However, this matrix only behaves like an identity when the matrix it's being multiplied is $n \times n$ (i.e. a square matrix).

\begin{definition}
	The $n \times n$ matrix $I_n$ is called the identity matrix and is defined by
	\begin{equation*}
		I_{i,j} = \begin{cases}
			1 & i=j \\
			0 & \text{otherwise}
		\end{cases}.
	\end{equation*}
\end{definition}

\begin{theorem}
	Let $A$ be an $n \times n$ matrix.
	Then $AI_n = I_n A = A$.
\end{theorem}
\begin{proof}
	Let $C = AI_n$.
	Notice,
	\begin{align*}
		C_{i,j} &= \sum_{d=1}^{n}{A_{i,d}(I_{n})_{d,j}} \\
			&= \sum_{d=1}^{n}{A_{i,d} \begin{cases}
					1 & d=j \\
					0 & \text{otherwise}
			\end{cases}} \\
			&= \sum_{d=1}^{n}{\begin{cases}
				A_{i,j} & d=j \\
				0 & \text{otherwise}
			\end{cases}} \\
			&= A_{i,j}.
	\end{align*}
	This same line of reasoning works to also show that $I_nA = A$.
	Since all entries of $A$ and $C$ are equal, $C = AI_n = A$, as desired.
\end{proof}

Also similar to scalar multiplication, matrix multiplication is associative and distributive.
\begin{theorem}
	Let $A$, $B$, and $C$ be matrices.
	Let $k$ be a scalar.
	Then the following properties hold (assuming the matrices have the correct dimensions):
	\begin{itemize}
		\item \textbf{Associative}: $A(BC) = (AB)C$.
		\item \textbf{Distributive Over Matrix Multiplication}: $k(AB) = (kA)B = A(kB)$.
		\item \textbf{Left Distributive Over Addition}: $A(B + C) = AB + AC$.
		\item \textbf{Right Distributive Over Addition}: $(A + B)C = AC + BC$.
	\end{itemize}
\end{theorem}

Unlike scalar multiplication, matrix multiplication is not commutative.
For one, if $AB$ is defined, $BA$ won't also be defined unless $A$ and $B$ are both square matrices
Even if this is the case, $AB \neq BA$ in general.

\begin{theorem}
	An $n \times n$ matrix $A$ commutes only and all matrices in the vector space $\linspan(\{I_n, A\})$.
\end{theorem}
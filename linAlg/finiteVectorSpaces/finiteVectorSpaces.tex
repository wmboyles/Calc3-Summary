\chapter{Finite Dimensional Vector Spaces}

If you've worked with vectors in $\R^2$ before, you might have noticed that we can write any vector using only two vectors and scalar multiplication: one to represent the $x$ component and another to represent the $y$ component.
\begin{equation*}
	\begin{bmatrix}
		v_x \\ v_y
	\end{bmatrix} = v_x \begin{bmatrix}
		1 \\ 0
	\end{bmatrix} + v_y \begin{bmatrix}
		0 \\ 1
	\end{bmatrix}.
\end{equation*}
Although $\langle 1, 0 \rangle$ and $\langle 0, 1 \rangle$ are probably the most convenient vectors to use to break up any vector, other representations are possible.
\begin{equation*}
	\begin{bmatrix}
		v_x \\ v_y
	\end{bmatrix} = \frac{v_y + v_x}{2} \begin{bmatrix}
		1 \\ 1
	\end{bmatrix} + \frac{v_y - v_x}{2} \begin{bmatrix}
		-1 \\ 1
	\end{bmatrix}.
\end{equation*}
We'll formalize for vector spaces like $\R^2$ when one can or can't write a vectors in this way, and how many vectors are needed to do so.

\section{Linear Span}

\subsection{Definition}
\begin{definition}
	Let $S = \{\vec{v_1}, \vec{v_2}, \dots, \vec{v_n}\}$ be a set of $n$ vectors from a vector space $V$.
	A linear combination of vectors from $S$ is any vector $\vec{v}$ such that
	\begin{equation*}
		\vec{v} = a_1\vec{v_1} + a_2\vec{v_2} + \dots + a_n\vec{v_n} = \sum_{i=1}^{n}{a_i\vec{v_i}},
	\end{equation*}
	where each $a_i$ is from the field associated with $V$.
\end{definition}

From above we see that any vectors in $\R^2$ can be written as a linear combination of the vectors $\langle 1, 0 \rangle$ and $\langle 0, 1 \rangle$.

\begin{definition}
	Let $S = \{\vec{v_1}, \vec{v_2}, \dots, \vec{v_n}\}$ be a set of $n$ vectors from a vector space $V$.
	The span of $S$, written $\linspan{S}$, is the set of all vectors from $V$ that can be written as a linear combination of vectors from $S$.
\end{definition}

\begin{example}
	\begin{equation*}
		\begin{bmatrix}
			1 \\ 3 \\ 5
		\end{bmatrix} \in \linspan{\left\{\begin{bmatrix}
				1 \\ 0 \\ 0
			\end{bmatrix}, \begin{bmatrix}
				0 \\ 1 \\ 0	
			\end{bmatrix}, \begin{bmatrix}
				0 \\ 0 \\ 1
			\end{bmatrix}\right\}}= \R^3 \text{; } 2x^2 - 3x + 3 \in \linspan\{2,1-x,1+x^2\}.
	\end{equation*}
\end{example}
\begin{example}
	\begin{equation*}
		\linspan\left\{\begin{bmatrix}
			1 \\ 2 \\ 3
		\end{bmatrix}, \begin{bmatrix}
			2 \\ 0 \\ 1
		\end{bmatrix}\right\} = \left\{\begin{bmatrix}
			x \\ y \\ z
		\end{bmatrix} \in \R^3 \mid 2x + 5y - 4z = 0\right\}
	\end{equation*}
\end{example}
\subsection{Span Is a Subspace}
\begin{theorem}
	Let $S = \{\vec{v_1}, \vec{v_2}, \dots, \vec{v_n}\}$ be a set of $n$ vectors from a vector space $V$.
	Then $\linspan{S}$ is the smallest subspace of $V$ containing all vectors in $S$.
\end{theorem}
\begin{proof}
	Since $S \subseteq V$, and $V$ is closed under additional and scalar multiplication by virtue of being a vector space, $\linspan{S} \subseteq V$.
	
	First we'll show closure under addition.
	Let $\vec{a}, \vec{b} \in \linspan{S}$.
	Then
	\begin{align*}
		\vec{a} &= a_1\vec{v_1} + a_2\vec{v_2} + \dots + a_n\vec{v_n} \\
		\vec{b} &= b_1\vec{v_1} + b_2\vec{v_2} + \dots + b_n\vec{v_n}.
	\end{align*}
	So,
	\begin{equation*}
		\vec{a} + \vec{b} = (a_1 + b_1)\vec{v_1} + (a_2 + b_2)\vec{v_2} + \dots + (a_n + b_n)\vec{v_n}.
	\end{equation*}
	Since each $(a_i + b_i)$ is in the associated field, we see that we can write $\vec{a} + \vec{b}$ as a linear combination of vectors from $S$.
	Thus, $\linspan{S}$ is closed under addition.
	
	Now we'll show scalar multiplication.
	Let $\vec{a} \in \linspan{S}$ and $k$ be an element from the associated field.
	\begin{equation*}
		k\vec{a} = (ka_1)\vec{v_1} + (ka_2)\vec{v_2} + \dots + (ka_n)\vec{v_n}.
	\end{equation*}
	Since each $ka_i$ is in the associated field, we see that we can write $k\vec{a}$ as a linear combination of vectors from $S$.
	Thus, $\linspan{S}$ is closed under multiplication.
	
	Now that we've show $\linspan{S}$ is a subspace of $V$, we'll show its the smallest one containing all elements from $S$.
	Suppose $U$ is a subspace of $V$ containing all elements from $S$.
	Let $\vec{a} \in \linspan{S}$.
	Since $U$ contains all vectors from $S$ and is a subspace, $\vec{a} \in U$.
	Thus, $\linspan{S} \subseteq U$.
\end{proof}
\subsection{Finite \& Infinite Dimension}
\begin{definition}
	Let $S = \{\vec{v_1}, \vec{v_2}, \dots, \vec{v_n}\}$ be a set of $n$ vectors.
	If $\linspan{S} = V$, a vector space, then we say
	\begin{itemize}
		\item $S$ spans $V$
		\item $S$ is a spanning set of $V$
	\end{itemize}
\end{definition}

\begin{definition}
	A vector space is finite dimensional if it has a finite spanning set.
	Otherwise, it is infinite dimensional.
\end{definition}

\begin{example}
	We see from the previous examples that
	\begin{equation*}
		\linspan{\left\{\begin{bmatrix}
				1 \\ 0 \\ 0
			\end{bmatrix}, \begin{bmatrix}
				0 \\ 1 \\ 0	
			\end{bmatrix}, \begin{bmatrix}
				0 \\ 0 \\ 1
			\end{bmatrix}\right\}}= \R^3.
	\end{equation*}
	Thus, $\R^3$ is a finite dimensional vector space.
\end{example}
\begin{example}
	The set of all polynomials with real coefficients is infinite dimensional because any finite set could only represent polynomials up to some finite degree.
\end{example}

\section{Linear Independence}

\subsection{Definition}

\begin{definition}
	Let $S = \{\vec{v_1}, \vec{v_2}, \dots, \vec{v_n}\}$ be a set of $n$ vectors from a vector space $V$. 
	The set $S$ is said to be linearly independent if the only solution to
	\begin{equation*}
		a_1\vec{v_1} + a_2\vec{v_2} + \dots + a_n\vec{v_n} = \vec{0}
	\end{equation*}
	is $a_1 = a_2 = \dots = a_n = 0$.
	Otherwise, $S$ is said to be linearly dependent.
\end{definition}

In other words, if we can write a vector from $S$ as a linear combination of the other vectors from $S$, then $S$ is linearly dependent.
Linearly independent sets don't have such redundancies.
Equivalently, $S$ is linearly independent if and only if every element of $\linspan{S}$ can be written in one and only way as a linear combination of elements from $S$.
If there were two unique representations, we could subtract them to get a non-zero solution to our equation.
Notice too that this equation itself is asking for a representation of $\vec{0}$ other than all coefficients set to 0.

\begin{example}
	Determine whether or not the following sets are linearly independent or dependent
	\begin{enumerate}
		\item 
		\begin{equation*}
			\left\{1,x,x^2, 2x^2 - x\right\}
		\end{equation*}
		\item
		\begin{equation*}
			\left\{\begin{bmatrix}
				1 \\ 0 \\ 1 \\ 2
			\end{bmatrix}, \begin{bmatrix}
				0 \\ 1 \\ 1 \\ 2
			\end{bmatrix}, \begin{bmatrix}
				1 \\ 1 \\ 1 \\ 3
			\end{bmatrix}\right\}
		\end{equation*}
	\end{enumerate}
\end{example}
\begin{answer}
	\begin{enumerate}
		\item
		This set is not linearly independent because
		\begin{equation*}
			0(1) + -1(x) + 2(x^2) + 1(2x^2 - x) = 0.
		\end{equation*}
		\item
		Let the three vectors be called $\vec{v_1}$, $\vec{v_2}$, and $\vec{v_3}$ respectively.
		Let's consider solutions ot the equation 
		\begin{equation*}
			a\vec{v_1} + b\vec{v_2} + c\vec{v_3} = \vec{0}.
		\end{equation*}
		Since $\vec{v_1}$ and $\vec{v_3}$, both have 1 in their first component, $a+c = 0$.
		Similarly, looking at $\vec{v_2}$ and $\vec{v_3}$, we find that $b+c = 0$.
		Thus, $a = b$ and $c = -a$.
		So, our equation becomes
		\begin{align*}
			a\vec{v_1} + a\vec{v_2} - a\vec{v_3} &= \vec{0} \\
			a\left(\vec{v_1} + \vec{v_2} - \vec{v_3}\right) &= \vec{0} \\
			a\begin{bmatrix}
				0 \\ 0 \\ 1 \\ 1
			\end{bmatrix} &= \vec{0} \\
			a &= 0.
		\end{align*}
		We see that the only solution is $a=b=c=0$, so the set is linearly independent.
	\end{enumerate}
\end{answer}
\subsection{Properties}
\begin{theorem}
	Let $V$ be a vector space.
	Let $S \subseteq T \subseteq V$.
	If $T$ is linearly independent, then so is $S$.
\end{theorem}
Notice that since all sets that are not linearly independent are linearly dependent, the contrapositive would be ``If $S$ is linearly dependent, then so is $T$''.

\begin{theorem}
	Let $S = \{\vec{v_1}, \vec{v_2}, \dots, \vec{v_n}\}$ be a set of $n$ vectors from a vector space $V$. 
	The set $S$ is linearly dependent if and only if there exists some $\vec{v_j} \in S$ such that $\vec{v_j} \in \linspan{(S \setminus \{\vec{v_j}\})}$.
	If this is the case, then $\linspan{S} = \linspan{(S \setminus \{\vec{v_j}\})}$.
\end{theorem}

\begin{theorem}
	Let $V$ be finite dimensional vector space.
	The size of any linearly independent set in $V$ is at most the size of a spanning set of $V$.
\end{theorem}
\begin{proof}
	Suppose $U = \{\vec{u_1}, \dots, \vec{v_m}\}$ is linearly independent in $V$.
	Suppose that $W = \{\vec{w_1}, \dots, \vec{w_n}\}$ spans $V$.
	
	Initialize $B = W$, then add $\vec{u_1}$ to $B$
	Since $W$ spans $V$, adding $\vec{u_1}$ to $B$ ensures that $B$ is linearly dependent.
	By previous result, we can remove some vector other than $\vec{u_1}$ from $B$ while still ensuring that $B$ spans $V$.
	Remove one such vector. \\
	
	Continue in this way, adding each vector from $U$ to $B$ one at a time.
	By previous result, we know we can remove one vector from $B$ while still being sure it will span $S$.
	Since all the vectors from $U$ are linearly independent, we can be sure that this vector will be from $W$. \\
	
	After doing this for all vectors in $U$, we see that at each step we were sure there was a vector from $W$ to remove.
	Thus, there must at least as many elements of $W$ than of $U$.
\end{proof}

\begin{theorem}
	Every subspace of a finite dimensional vector space is finite dimensional.
\end{theorem}
\begin{proof}
	Suppose $U$ is a subspace of $V$ where $V$ is finite dimensional.
	If $U = \{\vec{0}\}$, then we are done; otherwise, select a vector $\vec{v_1} \in U$.
	Continue in the following way: If $U = \linspan\{\vec{v_1}, \vec{v_2}, ... \vec{v_{j-1}}\}$ then we are done; otherwise, select a vector $\vec{v_j}$ not in this span.
	Since no vector we selected is in the span of the previous, the set of selected vectors is independent.
	By previous result, the set of selected vectors cannot be longer than a spanning set of $V$, so we know this process terminates.
	Thus, the set of selected vectors spans $U$ and is finite, meaning that $U$ is finite dimensional.
\end{proof}
\section{Bases}

\subsection{Definition}
\begin{definition}
	Let $V$ be a vector space.
	A non-empty set $S$ of vectors from $V$ is a basis for $V$ if
	\begin{enumerate}
		\item $S$ is a spanning set of $V$.
		\item $S$ is linearly independent.
	\end{enumerate}
\end{definition}

\begin{example}
	We saw before that the set $\{\langle 1, 0, 0 \rangle, \langle 0, 1, 0 \rangle, \langle 0, 0, 1 \rangle\}$ is linearly independent and spans $\R^3$.
	Thus, it is a basis for $\R^3$.
	This is called the ``standard basis'' for $\R^3$, and the same pattern holds for $\R^n$.
\end{example}
\begin{example}
	The set $\{1, x, x^2, \dots\}$ is a basis for the infinite dimensional vector space of polynomials with real coefficients.
\end{example}
\begin{example}
	Although $\{\langle 1, 0, 1, 2 \rangle, \langle 0, 1, 1, 2 \rangle, \langle 1, 1, 1, 3 \rangle \}$ is linearly independent, it does not span $\R^4$ because $\langle 1, 1, 1, 1 \rangle$ is in $\R^4$ but is not in the span of these three vectors.
\end{example}
\subsection{Properties}
A basis extends the idea of linear independence from just considering the zero vector to all vectors in a vector space.
\begin{theorem}
	A set of vectors $\{\vec{v_1}, \vec{v_2}, \dots, \vec{v_n}\}$ from a vector space $V$ is a basis for $V$ if and only every vector $\vec{v} \in V$ can be uniquely written in the form
	\begin{equation*}
		\vec{v} = a_1\vec{v_1} + a_2\vec{v_2} + \dots + a_n\vec{v_n},
	\end{equation*}
	where $a_1, a_2, \dots, a_n$ are from the field associated with $V$.
\end{theorem}

\begin{theorem}
	Every spanning set of a vector space $V$ can be made into a basis for $V$ by removing some elements.
\end{theorem}
\begin{proof}
	To make a spanning set into a basis, we need to remove vectors from the set to make it linearly independent, while still maintaining a spanning set of $V$.
	Start by removing all zero vectors from the set.
	Next, continue by removing vector $\vec{v_j}$ from the set if it's in $\linspan\{\vec{v_1}, \dots, \vec{v_{j-1}}\}$.
	The resulting set is linearly independent because all vectors are not in the span of the previous.
	The resulting set also still spans $V$ because we only removed vectors that were already in the span of previous vectors.
	Thus, the resulting set is a basis for $V$.
\end{proof}

\begin{corollary}
	Every finite dimensional vector space has a basis.
\end{corollary}
\begin{proof}
	By definition, a finite dimensional vector space has a finite spanning set.
	So, we simply apply the process described previously to make a basis from the spanning set.
\end{proof}

Just like we could remove vectors from a spanning set to form a basis, we can also add vectors to a linearly independent set.
\begin{theorem}
	Every linearly independent set of vectors from a finite dimensional vector space $V$ can be extend to make a basis for $V$.
\end{theorem}
\begin{proof}
	Let $U = \{\vec{u_1}, \vec{u_2}, \dots, \vec{v_m}\}$ be a linearly independent set of vectors from $V$.
	Let $W = \{\vec{w_1}, \vec{w_2}, \dots, \vec{w_n}\}$ be a basis of $V$.
	Then the set $U \cup W = \{\vec{u_1}, \dots, \vec{u_m}, \vec{w_1}, \dots \vec{w_n}\}$ certainly spans $V$.
	When we apply the procedure to turn this spanning set into a basis for $V$, none of the vectors from $U$ get removed, since they are linearly independent.
	Thus, our resulting basis is an extension of $U$ with some vectors from $W$, as desired.
\end{proof}
\section{Dimension}
We saw that a basis is both a linearly independent and spanning set.
We shall see that we can't have two bases for the same vector space of different sizes.
Thus, the length of a basis for a vector space is a special number, which we'll call ``dimension''.

\subsection{Definition}
\begin{theorem}
	Any two bases of the same finite dimensional vector space $V$ have the same size.
\end{theorem}
\begin{proof}
	Let $B_1$ and $B_2$ be two bases for $V$.
	Since $B_1$ is linearly independent in $V$, and $B_2$ spans $V$,
	\begin{equation*}
		\abs{B_1} \leq \abs{B_2},
	\end{equation*}
	by previous result.
	Similarly, since $B_2$ is linearly independent in $V$, and $B_1$ spans $V$,
	\begin{equation*}
		\abs{B_1} \leq \abs{B_2},
	\end{equation*}
	by previous result.
	Thus,
	\begin{equation*}
		\abs{B_1} = \abs{B_2},
	\end{equation*}
	as desired.
\end{proof}

\begin{definition}
	Let $V$ be a finite dimensional vector space.
	The dimension of $V$, denoted $\dim{V}$, is the size of any basis for $V$.
\end{definition}

\begin{example}
	Here are the dimension of some vector spaces:
	\begin{itemize}
		\item
		$\dim{\R^n} = n$, since the standard basis has $n$ elements.
		\item
		$\mathcal{P}_m$, the set of a polynomials with real coefficients of degree at most $m$, has dimension $m+1$ since $\{1, x, x^2, \dots, x^m\}$ is a basis.
		\item
		The set of all $n \times m$ matrices with real coefficients has dimension $nm$ (imagine ``unwrapping'' the matrix to look like a vector in $\R^{nm}$).
	\end{itemize}
\end{example}

\subsection{Properties}
Our results about the length of linearly independent sets and subspaces extend to dimension.
\begin{theorem}
	If $U$ is a subspace of a finite dimensional vector space $V$, then $\dim{U} \leq \dim{V}$.
\end{theorem}
\begin{proof}
	A basis for $U$ is linearly independent in $V$, and a basis for $V$ spans $V$.
	So, applying a previous result and the definition of dimension, $\dim{U} \leq \dim{V}$.
\end{proof}

Any linearly independent set of vectors with size equal to the dimension is a basis.
\begin{theorem}
	Let $V$ be a finite dimensional vector space.
	Then every linearly independent set of vectors from $V$ with size $\dim{V}$ is a basis for $V$.
\end{theorem}
\begin{proof}
	Let $\dim{V} = n$, and suppose $U = \{\vec{v_1}, \vec{v_2}, \dots, \vec{v_n}\}$ is linearly independent in $V$.
	We know by previous result that we can extend $U$ to form a basis.
	This basis must have $n$ elements.
	Thus, the extension is trivial and adds no vectors, meaning $U$ is a basis for $V$.
\end{proof}

Similarly, any spanning set of vectors with size equal to the dimension is a basis.
\begin{theorem}
	Let $V$ be a finite dimensional vector space.
	Then every spanning set of vectors from $V$ with size $\dim{V}$ is a basis for $V$
\end{theorem}
\begin{proof}
	Let $\dim{V} = n$, and suppose $S = \{\vec{v_1}, \dots, \vec{v_n}\}$ is a spanning set in $V$.
	We know by previous result that we can remove vectors from $S$ to form a basis.
	Thus basis must have $n$ elements.
	Thus, the removal is trivial and removes no vectors, meaning $S$ is a basis for $V$.
\end{proof}

\begin{example}
	Show that $\{\langle 1, 1 \rangle, \langle -1, 1 \rangle\}$ is a basis for $\R^2$.
\end{example}
\begin{answer}
	We showed previously that any vector from $\R^2$ can be written as a linear combination of these vectors.
	Thus, the set is spanning.
	Since $\dim{\R^2} = 2$, and our spanning set has 2 elements, it must be a basis.
\end{answer}

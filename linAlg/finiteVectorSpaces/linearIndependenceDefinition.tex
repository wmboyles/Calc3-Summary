\subsection{Definition}

\begin{definition}
	Let $S = \{\vec{v_1}, \vec{v_2}, \dots, \vec{v_n}\}$ be a set of $n$ vectors from a vector space $V$. 
	The set $S$ is said to be linearly independent if the only solution to
	\begin{equation*}
		a_1\vec{v_1} + a_2\vec{v_2} + \dots + a_n\vec{v_n} = \vec{0}
	\end{equation*}
	is $a_1 = a_2 = \dots = a_n = 0$.
	Otherwise, $S$ is said to be linearly dependent.
\end{definition}

In other words, if we can write a vector from $S$ as a linear combination of the other vectors from $S$, then $S$ is linearly dependent.
Linearly independent sets don't have such redundancies.
Equivalently, $S$ is linearly independent if and only if every element of $\linspan{S}$ can be written in one and only way as a linear combination of elements from $S$.
If there were two unique representations, we could subtract them to get a non-zero solution to our equation.
Notice too that this equation itself is asking for a representation of $\vec{0}$ other than all coefficients set to 0.

\begin{example}
	Determine whether or not the following sets are linearly independent or dependent
	\begin{enumerate}
		\item 
		\begin{equation*}
			\left\{1,x,x^2, 2x^2 - x\right\}
		\end{equation*}
		\item
		\begin{equation*}
			\left\{\begin{bmatrix}
				1 \\ 0 \\ 1 \\ 2
			\end{bmatrix}, \begin{bmatrix}
				0 \\ 1 \\ 1 \\ 2
			\end{bmatrix}, \begin{bmatrix}
				1 \\ 1 \\ 1 \\ 3
			\end{bmatrix}\right\}
		\end{equation*}
	\end{enumerate}
\end{example}
\begin{answer}
	\begin{enumerate}
		\item
		This set is not linearly independent because
		\begin{equation*}
			0(1) + -1(x) + 2(x^2) + 1(2x^2 - x) = 0.
		\end{equation*}
		\item
		Let the three vectors be called $\vec{v_1}$, $\vec{v_2}$, and $\vec{v_3}$ respectively.
		Let's consider solutions ot the equation 
		\begin{equation*}
			a\vec{v_1} + b\vec{v_2} + c\vec{v_3} = \vec{0}.
		\end{equation*}
		Since $\vec{v_1}$ and $\vec{v_3}$, both have 1 in their first component, $a+c = 0$.
		Similarly, looking at $\vec{v_2}$ and $\vec{v_3}$, we find that $b+c = 0$.
		Thus, $a = b$ and $c = -a$.
		So, our equation becomes
		\begin{align*}
			a\vec{v_1} + a\vec{v_2} - a\vec{v_3} &= \vec{0} \\
			a\left(\vec{v_1} + \vec{v_2} - \vec{v_3}\right) &= \vec{0} \\
			a\begin{bmatrix}
				0 \\ 0 \\ 1 \\ 1
			\end{bmatrix} &= \vec{0} \\
			a &= 0.
		\end{align*}
		We see that the only solution is $a=b=c=0$, so the set is linearly independent.
	\end{enumerate}
\end{answer}
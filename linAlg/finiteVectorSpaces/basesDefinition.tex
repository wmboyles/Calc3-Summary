\subsection{Definition}
\begin{definition}
	Let $V$ be a vector space.
	A non-empty set $S$ of vectors from $V$ is a basis for $V$ if
	\begin{enumerate}
		\item $S$ is a spanning set of $V$.
		\item $S$ is linearly independent.
	\end{enumerate}
\end{definition}

\begin{example}
	We saw before that the set $\{\langle 1, 0, 0 \rangle, \langle 0, 1, 0 \rangle, \langle 0, 0, 1 \rangle\}$ is linearly independent and spans $\R^3$.
	Thus, it is a basis for $\R^3$.
	This is called the ``standard basis'' for $\R^3$, and the same pattern holds for $\R^n$.
\end{example}
\begin{example}
	The set $\{1, x, x^2, \dots\}$ is a basis for the infinite dimensional vector space of polynomials with real coefficients.
\end{example}
\begin{example}
	Although $\{\langle 1, 0, 1, 2 \rangle, \langle 0, 1, 1, 2 \rangle, \langle 1, 1, 1, 3 \rangle \}$ is linearly independent, it does not span $\R^4$ because $\langle 1, 1, 1, 1 \rangle$ is in $\R^4$ but is not in the span of these three vectors.
\end{example}
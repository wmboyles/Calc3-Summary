\subsection{Properties}
\begin{theorem}
	Let $V$ be a vector space.
	Let $S \subseteq T \subseteq V$.
	If $T$ is linearly independent, then so is $S$.
\end{theorem}
Notice that since all sets that are not linearly independent are linearly dependent, the contrapositive would be ``If $S$ is linearly dependent, then so is $T$''.

\begin{theorem}
	Let $S = \{\vec{v_1}, \vec{v_2}, \dots, \vec{v_n}\}$ be a set of $n$ vectors from a vector space $V$. 
	The set $S$ is linearly dependent if and only if there exists some $\vec{v_j} \in S$ such that $\vec{v_j} \in \linspan{(S \setminus \{\vec{v_j}\})}$.
	If this is the case, then $\linspan{S} = \linspan{(S \setminus \{\vec{v_j}\})}$.
\end{theorem}

\begin{theorem}
	Let $V$ be finite dimensional vector space.
	The size of any linearly independent set in $V$ is at most the size of a spanning set of $V$.
\end{theorem}
\begin{proof}
	Suppose $U = \{\vec{u_1}, \dots, \vec{v_m}\}$ is linearly independent in $V$.
	Suppose that $W = \{\vec{w_1}, \dots, \vec{w_n}\}$ spans $V$.
	
	Initialize $B = W$, then add $\vec{u_1}$ to $B$
	Since $W$ spans $V$, adding $\vec{u_1}$ to $B$ ensures that $B$ is linearly dependent.
	By previous result, we can remove some vector other than $\vec{u_1}$ from $B$ while still ensuring that $B$ spans $V$.
	Remove one such vector. \\
	
	Continue in this way, adding each vector from $U$ to $B$ one at a time.
	By previous result, we know we can remove one vector from $B$ while still being sure it will span $S$.
	Since all the vectors from $U$ are linearly independent, we can be sure that this vector will be from $W$. \\
	
	After doing this for all vectors in $U$, we see that at each step we were sure there was a vector from $W$ to remove.
	Thus, there must at least as many elements of $W$ than of $U$.
\end{proof}

\begin{theorem}
	Every subspace of a finite dimensional vector space is finite dimensional.
\end{theorem}
\begin{proof}
	Suppose $U$ is a subspace of $V$ where $V$ is finite dimensional.
	If $U = \{\vec{0}\}$, then we are done; otherwise, select a vector $\vec{v_1} \in U$.
	Continue in the following way: If $U = \linspan\{\vec{v_1}, \vec{v_2}, ... \vec{v_{j-1}}\}$ then we are done; otherwise, select a vector $\vec{v_j}$ not in this span.
	Since no vector we selected is in the span of the previous, the set of selected vectors is independent.
	By previous result, the set of selected vectors cannot be longer than a spanning set of $V$, so we know this process terminates.
	Thus, the set of selected vectors spans $U$ and is finite, meaning that $U$ is finite dimensional.
\end{proof}
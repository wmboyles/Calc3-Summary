\subsection{Definition}
\begin{theorem}
	Any two bases of the same finite dimensional vector space $V$ have the same size.
\end{theorem}
\begin{proof}
	Let $B_1$ and $B_2$ be two bases for $V$.
	Since $B_1$ is linearly independent in $V$, and $B_2$ spans $V$,
	\begin{equation*}
		\abs{B_1} \leq \abs{B_2},
	\end{equation*}
	by previous result.
	Similarly, since $B_2$ is linearly independent in $V$, and $B_1$ spans $V$,
	\begin{equation*}
		\abs{B_1} \leq \abs{B_2},
	\end{equation*}
	by previous result.
	Thus,
	\begin{equation*}
		\abs{B_1} = \abs{B_2},
	\end{equation*}
	as desired.
\end{proof}

\begin{definition}
	Let $V$ be a finite dimensional vector space.
	The dimension of $V$, denoted $\dim{V}$, is the size of any basis for $V$.
\end{definition}

\begin{example}
	Here are the dimension of some vector spaces:
	\begin{itemize}
		\item
		$\dim{\R^n} = n$, since the standard basis has $n$ elements.
		\item
		$\mathcal{P}_m$, the set of a polynomials with real coefficients of degree at most $m$, has dimension $m+1$ since $\{1, x, x^2, \dots, x^m\}$ is a basis.
		\item
		The set of all $n \times m$ matrices with real coefficients has dimension $nm$ (imagine ``unwrapping'' the matrix to look like a vector in $\R^{nm}$).
	\end{itemize}
\end{example}
\subsection{Invertible Functions}

\subsubsection{Generally}
\begin{definition}
	A function $T: V \to W$ is invertible if there exists an inverse function $T^{-1}: W \to V$ such that $T \circ T^{-1}$ is the identity on $W$ and $T^{-1} \circ T$ is the identity on $V$.
\end{definition}
That is, invertible functions are those that are reversible both ways.
Notice too that $T$ is the inverse of $T^{-1}$.
That is, $(T^{-1})^{-1} = T$.

\begin{theorem}
	The inverse of function is unique.
\end{theorem}
\begin{proof}
	Let $T: V \to W$ be an invertible function.
	Let $S_1, S_2: W \to V$ be inverses of $T$.
	Then
	\begin{equation*}
		S_1 = S_1 \circ I = S_1 \circ (T \circ S_2) = (S_1 \circ T) \circ S_2 = I \circ S_2 = S_2.
	\end{equation*}
	Thus, all inverses of $T$ are in fact the same function, as desired.
\end{proof}
This uniqueness property is what allows the notation $T^{-1}$ be be well-defined.

\begin{example}
	Find the inverse function of the $T: \R^3 \to \R^3$ defined by
	\begin{equation*}
		T\begin{bmatrix}
			x \\ y \\ z
		\end{bmatrix} = \frac{1}{2}\begin{bmatrix}
			x+y \\ y+z \\ z+x
		\end{bmatrix}.
	\end{equation*}
\end{example}
\begin{answer}
	We'll check that the function $S: \R^3 \to \R^3$ defined by
	\begin{equation*}
		S\begin{bmatrix}
			x \\ y \\ z
		\end{bmatrix} = \begin{bmatrix}
			x-y+z \\ x+y-z \\ -x+y+z
		\end{bmatrix}
	\end{equation*}
	is the inverse.
	\begin{alignat*}{2}
		(T \circ S)\left(\begin{bmatrix} x \\ y \\ z\end{bmatrix}\right)
		&= T\left(\begin{bmatrix} x-y+z \\ x+y-z \\ -x+y+z \end{bmatrix}\right) 
		&= \frac{1}{2} \begin{bmatrix} (x-y+z)+(x+y-z) \\ (x+y-z)+(-x+y+z) \\ (-x+y+z)+(x-y+z)\end{bmatrix} 
		&= \begin{bmatrix} x \\ y \\ z \end{bmatrix}. \\
		(S \circ T)\left(\begin{bmatrix} x \\ y \\ z \end{bmatrix}\right)
		&= S\left(\frac{1}{2}\begin{bmatrix} x+y \\ y+z \\ z+x \end{bmatrix}\right) 
		&= \frac{1}{2}\begin{bmatrix} (x+y)-(y+z)+(z+x) \\ (x+y)+(y+z)-(z+x) \\ -(x+y)+(y+z)+(z+x) \end{bmatrix} 
		&= \begin{bmatrix} x \\ y \\ z \end{bmatrix}.
	\end{alignat*}
	Since $T \circ S$ and $S \circ T$ are both the identity, $S$ is indeed the inverse of $T$.
\end{answer}

\begin{definition}
	If a function $T: V \to W$ is both injective and surjective, then it is bijective.
	We say then that $T$ is a bijection.
\end{definition}

\begin{theorem}
	A function is invertible if and only if it is bijective.
\end{theorem}
\begin{proof}
	Let $T: V \to W$ be a function. \\
	
	$\implies$: Suppose $T$ is invertible.
	We'll show that $T$ is a bijection by showing it's both injective in surjective.
	Let $u, v \in V$ such taht $T(u) = T(v)$.
	Then
	\begin{equation*}
		u = T^{-1}(T(u)) = T^{-1}(T(v)) = v.
	\end{equation*}
	So, $T$ is injective.
	Let $w \in W$.
	Since $T$ is invertible, $w = T(T^{-1}(w))$.
	Hence, $w \in \range{T}$, meaning $\range{T} = W$, so $T$ is surjective. \\
	
	$\impliedby$: Suppose $T$ is a bijection.
	For each element $w \in W$, define $S(w)$ to be the unique element of $V$ such that $(T \circ S)(w) = w$.
	The existence of $S$ and uniqueness of its output are implied by $T$ being a bijection.
	We see that by construction, $T \circ S$ is the identity on $W$.
	So, all that remains is to show that $S \circ T$ is the identity on $V$.
	Let $v \in V$.
	Then
	\begin{equation*}
		T((S \circ T)(v)) = (T \circ S)(T(v)) = I(T(v)) = T(v).
	\end{equation*}
	Since $T$ is injective, this result implies that $(S \circ T)(v) = v$, meaning $S \circ T$ is the identity on $V$.
\end{proof}

\subsubsection{For Linear Maps}
Inverse functions behave nicely when looking just at linear maps.
In particular, the inverse of a linear map is also a linear map.

\begin{theorem}
	The inverse of a linear map is a linear map.
\end{theorem}
\begin{proof}
	Let $T: U \to V$ be an invertible linear map.
	Let $x,y \in V$.
	Then
	\begin{align*}
		T^{-1}(x + y) &= T^{-1}\left(T(T^{-1}(x)) + T(T^{-1}(y))\right) \\
		&= T^{-1}(T(T^{-1}(x) + T^{-1}(y))) \\
		&= T^{-1}(x) + T^{-1}(y).
	\end{align*}
	Let $k$ be from the field associated with $U$ and $V$.
	Then
	\begin{align*}
		T^{-1}(kx) &= T^{-1}(k T(T^{-1}(T(x))) ) \\
		&= T^{-1}(T(k T^{-1}(x))) \\
		&= kT^{-1}(x).
	\end{align*}
	Thus, $T^{-1}$ is a linear map.
\end{proof}

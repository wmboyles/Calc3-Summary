\subsection{Properties}

\begin{theorem}
	A linear transformation $T: V \to W$ is uniquely determined by how is transforms the vectors in a basis for $V$.
\end{theorem}
\begin{proof}[Proof Sketch]
	Let $B = \{\vec{v_1}, \vec{v_2}, \dots, \vec{v_n}\}$ be a basis for $V$.
	Let $\vec{v} \in V$.
	Since $B$ is a basis, $\vec{v}$ has a unique representation as a linear combination of vectors from $B$.
	So,
	\begin{align*}
		\vec{v} &= a_1\vec{v_1} + a_2\vec{v_2} + \dots + a_n\vec{v_n} \\
		T\vec{v} &= T(a_1\vec{v_1} + a_2\vec{v_2} + \dots + a_n\vec{v_n}) \\
			&= T(a_1\vec{v_1}) + T(a_2\vec{v_2}) + \dots + T(a_n\vec{v_n}) \\
			&= a_1T\vec{v_1} + a_2T\vec{v_2} + \dots + a_nT\vec{v_n}.
	\end{align*}
\end{proof}

\begin{corollary}
	For any linear transformation $T: V \to W$, $T\vec{0} = \vec{0}$.
\end{corollary}

\begin{theorem}
	Let $B = \{\vec{v_1}, \vec{v_2}, \dots, \vec{v_n}\}$ be a basis for $V$ and $\vec{w_1},\dots,\vec{w_n} \in W$.
	Then there exists a unique linear map $T: V \to W$ such that
	\begin{equation*}
		T\vec{v_j} = \vec{w_j}
	\end{equation*}
	for $j = 1, 2, \dots, n$.
	In particular, the transformation is
	\begin{equation*}
		T(a_1\vec{v_1} + \dots + a_n\vec{v_n}) = a_1\vec{w_1} + \dots + a_n\vec{w_n}.
	\end{equation*}
\end{theorem}

\subsubsection{Operations on Set of Linear Transformations}

We can define addition and scalar multiplication on $\mathcal{L}(V,W)$ and show that we in fact have a vector space.
\begin{definition}
	Let $S, T \in \mathcal{L}(V,W)$ and $k$ be a scalar from the associated field.
	Then for all $\vec{v} \in V$,
	\begin{equation*}
		(S+T)(\vec{v}) = S\vec{v} + T\vec{v}
	\end{equation*}
	and
	\begin{equation*}
		(kT)(\vec{v}) = k(T\vec{v}).
	\end{equation*}
\end{definition}

\begin{theorem}
	The set $\mathcal{L}(V,W)$ with the operations defined above is a vector space.
\end{theorem}

We can also define the multiplication of two elements as function composition.
\begin{definition}
	Let $S \in \mathcal{L}(V,U)$ and $T \in \mathcal{L}(U,W)$.
	Then for all $\vec{v} \in V$,
	\begin{equation*}
		(ST)(\vec{v}) = S(T\vec{v}).
	\end{equation*}
\end{definition}
This is normally notated $ST\vec{v}$.
Notice that even if the domains and codomains allow $TS\vec{v}$ to be defined, multiplication of linear transformations is not in general commutative: $ST\vec{v} \neq TS\vec{v}$.

There are however some properties of multiplication that hold
\begin{theorem}
	The following properties of multiplication of linear transformations holds (assuming the domains and codomains allow the operations to be defined):
	\begin{itemize}
		\item \textbf{Associativity}: $(T_1T_2)T_3 = T_1(T_2T_3)$.
		\item \textbf{Identity}: $TI = IT = T$.
		\item \textbf{Left \& Right Distributive}: $(T_1 + T_2)T_3 = T_1T_3 + T_2T_3$ and $T_1(T_2 + T_3) = T_1T_2 + T_1T_3$.
	\end{itemize}
\end{theorem}
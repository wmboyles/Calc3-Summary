\subsection{Definition \& Notations}
\begin{definition}
	A Euclidean vectors is a finite, ordered list of real numbers.
	Each entry in the vector is called a component.
	The number of components in a vector is called its dimension.
	The set of all Euclidean vectors with dimension $n$ is notated $\R^n$ and called ``Euclidean $n$-space''.
\end{definition}

Often, vectors are written in square brackets with their components ordered vertically from top to bottom.
We tend to symbolically represent some vector by a symbol with an arrow above it.
We can represent a certain component of a vector with a subscript.

\begin{example}
	The vector $\vec{v} = \begin{bmatrix}4 \\ 5\end{bmatrix}$ is a vector in $\R^2$.
	The components are $\vec{v}_1 = 4$ and $\vec{v}_2 = 5$.
	Since $\vec{v}$ can be visualized as a straight arrow in the 2D plane with its tail at the origin and its tip at $(4,5)$, we might also notate the $x$ and $y$ components as $\vec{v}_x = 4$ and $\vec{v}_y = 5$.
	
	\begin{figure}[H]
		\centering
		\includegraphics[scale=0.5]{../common/vectorsMatrices/VectorAddition.png}
		\caption{The $x$ and $y$ components of a vector $v$}
	\end{figure}
\end{example}
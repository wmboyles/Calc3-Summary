\subsection{Definition \& Notations}
\begin{definition}
	For some domain $D$, A binary operation is a function $f: D \times D \to D$.
	That is, it combines two elements of $D$ to produce another element of $D$.
\end{definition}

You've already been working with binary operations since elementary school.
\begin{example}
	The following are binary operations with domain $\R$.
	\begin{itemize}
		\item Addition ($+$)
		\item Multiplication ($\times$)
	\end{itemize}

	Subtraction ($-$) is also a binary operation on $\R$.
	It is just a special case of addition, since for all $a,b \in \R$, $a - b = a + (-b)$, and $-b \in \R$.
\end{example}

One needs to specify over which domain a function is a binary operation.
Functions can be binary operations over some domains but not others.
\begin{example}
	\hspace{1pt}
	\begin{itemize}
		\item
		Division ($\div$) is not a binary operation on $\R$ since $0 \in \R$, but dividing by 0 is not defined.
		However if we consider the domain $\R \setminus \{0\}$, then division is a binary operation, since we no longer have to worry about dividing by 0.
		\item
		Although we saw that subtraction was a binary operation over $\R$, it is not over $D = \{x \in \R \mid x \geq 0\}$ because $1,2 \in D$, but $1 - 2 = -1 \not\in D$.
	\end{itemize}
\end{example}

For many binary operations, we tend to abandon function notation in favor of an ``infix'' notation.
For example, we write $a + b$ or $a \times b$ rather than $+(a,b)$ or $\times(a,b)$.
Sometimes, if the operation is clear from context, we may drop the symbol for the operation entirely and just write the inputs next to each other.
For example, if $a,b \in \R$, then $ab$ is understood to mean $a \times b$.
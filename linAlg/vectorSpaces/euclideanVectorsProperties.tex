\subsection{Properties}

We can define addition and scalar multiplication of Euclidean vectors.

\begin{definition}
	Let $\vec{a},\vec{b} \in \R^n$ and $c \in \R$ then
	\begin{enumerate}
		\item
		\begin{equation*}
			\vec{a} + \vec{b} = \left[a_i + b_i \mid i \in \{1,2,\dots,n\}\right],
		\end{equation*}
		\item
		\begin{equation*}
			c \cdot \vec{a} = \left[c \cdot a_i \mid i \in \{1,2,\dots,n\}\right].
		\end{equation*}
	\end{enumerate}
\end{definition}

\begin{example}
	Let $\vec{u} = \begin{bmatrix} 1 \\ -2 \\ 3 \end{bmatrix}$,
	$\vec{v} = \begin{bmatrix} -1 \\ 4 \\ 3 \end{bmatrix}$, and
	$\vec{w} = \begin{bmatrix} 4 \\ 2 \\ 6 \end{bmatrix}$.
	Find $(2\vec{u} + \vec{v}) - 3\vec{w}$.
\end{example}
\begin{answer}
	We find
	\begin{align*}
		(2\vec{u} + \vec{v}) - 3\vec{w} &= \left(2\begin{bmatrix} 1 \\ -2 \\ 3 \end{bmatrix} + \begin{bmatrix} -1 \\ 4 \\ 3 \end{bmatrix}\right) - 3\begin{bmatrix} 4 \\ 2 \\ 6 \end{bmatrix} \\
		&= \left(\begin{bmatrix}2 \\ -4 \\ 6\end{bmatrix} + \begin{bmatrix} -1 \\ 4 \\ 3 \end{bmatrix}\right) + \begin{bmatrix}-12 \\ -6 \\ -18\end{bmatrix} \\
		&= \begin{bmatrix}1 \\ 0 \\ 9\end{bmatrix} + \begin{bmatrix}-12 \\ -6 \\ -18\end{bmatrix} \\
		&= \begin{bmatrix}-11 \\ -6 \\ -9\end{bmatrix}.
	\end{align*}
\end{answer}

Notice that addition of vectors is a binary operation over $\R^n$.
Also, many of the properties we said binary operations could have are satisfied.
\begin{theorem}
	The addition of two vectors in $\R^n$ is commutative and associative.
	The identity vector is the one containing all 0's.
	The inverse of any vector $\vec{v}$ is $-1 \cdot \vec{v} = -\vec{v}$.
	Vector-scalar multiplication\footnote{Vector-scalar multiplication is not a binary operation, but the distributive property still holds} is distributive over vector addition.
\end{theorem}


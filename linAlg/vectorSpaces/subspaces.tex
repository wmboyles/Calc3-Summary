\subsection{Subspaces}
\begin{definition}
	A non-empty subset $U$ of a vector space $V$ is a subspace of $V$ if $U$ is a vector space with the operations inherited from $V$.
\end{definition}

Here are a few examples of subspaces:
\begin{example}
	\hspace{1pt}
	\begin{itemize}
		\item 
			The set of vectors $\left\{\begin{bmatrix}x \\ y \\ 0\end{bmatrix} \mid x,y \in \R\right\}$ is a subspace of $\R^3$.
		\item
			The set of real-valued functions $\{f \in \R^{[0,1]} \mid \int_{0}^{1}{f(x)\d{x}} = 0\}$ is a subspace of all real-valued functions that are integrable over $[0,1]$, $\R^{[0,1]}$.
		\item
			The set of functions $\{f: [0,1] \to \R \mid f'(1/2) = 0\}$ is a subspace of $\R^{[0,1]}$.
	\end{itemize}
\end{example}

Here are a few non-examples of subspaces:
\begin{example}
	\hspace{1pt}
	\begin{itemize}
		\item
			The set of vectors $W = \left\{\begin{bmatrix}x\\y\\z\end{bmatrix} \mid x,y,z \in \R, 2x+y-z=-4\right\}$ is not a subspace of $\R^3$ because $\vec{0} \not\in W$, meaning there is no additive identity.
		\item
			The set of real-valued functions $\{f \in \R^{[0,1]} \mid \int_{0}^{1}{f(x)\d{x}} = 1\}$ is not a subspace of $\R^{[0,1]}$ because it is not closed under addition.
		\item
			The set of functions $\{f: [0,1] \to \R \mid f'(1/2) = 1\}$ is not a subspace of $\R^{[0,1]}$ because it is not closed under scalar multiplication.
	\end{itemize}
	
\end{example}

It would be tedious to recheck all the vector space conditions when seeing if a set is a subspace.
Thankfully, we have an easier subspace test
\begin{theorem}
	Let $U$ be a non-empty subset of a vector space $V$.
	$U$ is a subspace of $V$ if and only if $U$ is closed under addition and scalar multiplication.
\end{theorem}
\begin{proof}
	Assume that $U$ is closed under addition and scalar multiplication.
	Since $U \subseteq V$, the addition properties of commutativity and associativity hold, the multiplicative identity is in $U$, and the distributive properties all hold.
	Thus, all that remains to check is whether the additive inverse exists, and whether each element of $U$ has an additive inverse. \\
	
	Let $\vec{u} \in U$.
	Since $U \subset V$ and $V$ is a vector space, there exists $-\vec{u} \in V$.
	By previous result, we know that $-\vec{u} = -1\cdot\vec{u}$.
	Since $U$ is closed under scalar multiplication, $-\vec{u} \in U$.
	Further, since $\vec{u} + -\vec{u} = \vec{0}$, and $U$ is closed under addition, $\vec{0} \in U$.
\end{proof}
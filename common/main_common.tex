% This file contains all the items that are common to every main.tex.
% Ideally, a main.tex should only need to include this file and then begin the document.

\documentclass[oneside, 12pt]{book}

% ------------------------------------------------------------------------------
% Table of Contents

\setcounter{tocdepth}{3} 	    % TOC should label down to subsubsections
\setcounter{secnumdepth}{2}	    % TOC should not number further than a subsection number
\setcounter{chapter}{-1}        % Start with chapter 0
\usepackage[titletoc]{appendix} % Include optional appendix
% ------------------------------------------------------------------------------


% ------------------------------------------------------------------------------
% General

\usepackage[english]{babel}
\usepackage{amsmath,amsthm, amssymb,bm} % Formatting symbols, theorems, lemmas, definitions, and examples
\usepackage{mathspec}

\setmathsfont(Digits,Latin,Greek){FreeSerif}

\setmathrm{FreeSerif}

\setmainfont{FreeSerif}

\usepackage{float}							    % For making sure tables and figures stay in place
\usepackage{fullpage}						    % Create ~1 inch margins
\usepackage{changepage}							% For setting up indents after examples
\usepackage[bottom,flushmargin]{footmisc}	    % Put footnotes at bottom of page; don't intent footnotes
\setlength{\parindent}{0pt}                     % Don't indent paragraphs by default

\usepackage{enumitem} 					        % For listing items and sub-items
\usepackage{hyperref} 					        % For inserting links
\usepackage{graphicx} 					        % For inserting images
\usepackage{float}						        % For specifying image location
% ------------------------------------------------------------------------------


% ------------------------------------------------------------------------------
% Tikz

\usepackage{tikz}
\usetikzlibrary{patterns}
\usetikzlibrary{calc,patterns,decorations.pathmorphing,decorations.markings}
% ------------------------------------------------------------------------------


% ------------------------------------------------------------------------------
% Setup for specially named sections

% Don't number these items
\newtheorem*{theorem}{Theorem}
\newtheorem*{corollary}{Corollary}
\newtheorem*{definition}{Definition}
\newtheorem*{lemma}{Lemma}
\newtheorem*{example}{Example}

% Answers (usually below examples) are indented
\newenvironment{answer}{\begin{adjustwidth}{15pt}{}}{\end{adjustwidth}}
% ------------------------------------------------------------------------------


% ------------------------------------------------------------------------------
% Command Shortcuts

\renewcommand{\qedsymbol}{$\blacksquare$}                                           % QED symbol at end of proof is a black square

\renewcommand{\d}[1]{\mathrm{d} #1}                                                 % Use non-italicized d for differentials
\newcommand{\dd}[2]{\frac{\d{#1}}{\d{#2}}}                                          % d[] / d[]
\newcommand{\pp}[2]{\frac{\partial #1}{\partial #2}}								% partial[] / partial[]


\newcommand{\abs}[1]{\big\lvert #1 \big\rvert}                                      % Absolute value
\newcommand{\norm}[1]{\abs{\abs{#1}}} 	                                            % Double-bar for vector magnitude


\DeclareMathOperator{\arcsec}{arcsec}			                                    % arcsec
\DeclareMathOperator{\arccot}{arccot}			                                    % arccot
\DeclareMathOperator{\arccsc}{arccsc}			                                    % arccsc

\newcommand{\R}{\mathbb{R}}						                                    % Real numbers
\newcommand{\Z}{\mathbb{Z}}

\newcommand{\Laplace}[1]{\mathcal{L}\left\{ #1 \right\}\left(s\right)}				% Laplace Transform
\newcommand{\inverseLaplace}[1]{\mathcal{L}^{-1}\left\{ #1 \right\}\left(t\right)}	% Inverse Laplace Transform
% ------------------------------------------------------------------------------


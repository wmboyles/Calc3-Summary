\subsection{Exponentials \& Logarithms}
\begin{definition}
	e is the base of the natural logarithm. It's defined by the limit
	\begin{equation*}
		e = \lim\limits_{n\rightarrow\infty}{\left(1+\frac{1}{n}\right)^n}.
	\end{equation*}
\end{definition}

The functions $\exp{x} = e^x$ and $\ln{x}$ are inverses of each other:
\begin{equation*}
	e^{\ln{x}} = x \text{ and } \ln{e^x} = x.
\end{equation*}

Just like other exponentials, the normal rules for adding, subtracting, and multiplying exponents apply:
\begin{equation*}
	e^xe^y = e^{x+y} \text{, } \frac{e^x}{e^y}=e^{x-y} \text{, and } \left(e^x\right)^k=e^{xk}.
\end{equation*}

Similar rules apply for logarithms:
\begin{equation*}
	\ln{x}+\ln{y} = \ln{xy} \text{, } \ln{x}-\ln{y} = \ln{\left(\frac{x}{y}\right)} \text{, and } \ln{\left(a^b\right)} = b\ln{a}.
\end{equation*}

We can also write a logarithm of any base using natural logarithms:
\begin{equation*}
	\log_{b}{a} = \frac{\ln{a}}{\ln{b}}.
\end{equation*}

The number $e$ is also unique in that it is the only real number $a$ satisfying the equation
\begin{equation*}
	\frac{\mathrm{d}}{\mathrm{d}x}a^x = a^x,
\end{equation*}
meaning $e^x$ is its own derivative.